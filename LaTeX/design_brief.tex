\documentclass{cce2014-design}
\svnInfo $Id$

% Document details
\title{Design Brief: Group 5}
\author{
   Akram Sharara,
   Antoni Marek Skoć,
   Daniel Agius}
\date{March 13, 2025, Document v.1}

\usepackage{graphicx}

\begin{document}

\maketitle

\abstract{%
Abstract

As a record of the development of the Morse code receiver, this initial design brief provides an outline of the aims and objectives of the project as set out in the handbook and how they will be realized. It discusses a comprehensive view of the development stages of the receiver using Fig. 1 and how signal sampling will be handled to convert analog audio inputs to digital ones. Finally, it explores how current code implementations will use this data to translate Morse code to characters using a binary tree, in addition to functions that could provide a clear approach to scheduling and displaying the outputs on an LCD. A timeline is provided to document management of ongoing work as well as areas of focus. Although the timeline is flexible, work distribution is clear among members and a prototype receiver is expected to be prepared around Easter.
}

\section{Introduction}
The objective of this assignment is to design and implement a microcontroller-based system that receives an input Amplitude Shift Keyed (ASK) Morse code signal, translates it into characters, and displays it on an LCD. Additionally, our implementation of the Morse code translator may take into consideration the possible presence of noise in the signal, as well as adapt to different transmission speeds. All code will be written in C/C++. We will be utilizing the Embedded Artists LPC4088 microcontroller, which is based on the ARM Cortex-M4 architecture and provides the necessary peripherals for efficient signal processing and display.

\section{System Design}
\subsection{General Design}
The initial design plan will follow the outline shown in figure 1. Initially, the audio input is taken by the receiver as an analogue sinusoidal tone of a defined frequency. Once this signal is sampled, it can then be converted from an ASK signal to a digital signal. Using a binary tree, this can then be decoded to a particular character to be displayed on the LCD using the necessary display functions.

\begin{figure}[h]
    \centering
    \includegraphics[width=0.45\textwidth]{Untitled_Diagram.drawio.png}
    \caption{diagram outlining the development of the Morse code receiver.}
    \label{fig:stage_diagram}
\end{figure}

The LPC4088 microcontroller, with its integrated Analog-to-Digital Converter (ADC) and high-speed processing capabilities, is particularly well-suited for sampling the analog ASK signal and performing the necessary signal processing. The microcontroller's Direct Memory Access (DMA) feature will allow efficient data transfer from the ADC to memory, reducing processor load and enabling fast, real-time processing.

\subsection{Signal Sampling and Conversion to a Digital Signal}
The sampling and demodulation part is responsible for converting the amplitude shift keyed waveform into a binary high or low internal output, which is necessary for conversion into dots and dashes. The LPC4088’s ADC, capable of 12-bit resolution, will be used to sample the analogue input signal with high precision. The microcontroller can sample data at high rates, ensuring that the timing between samples is accurate.

For sampling, the receiver will use the analogue pins of the microcontroller to sample every \( T \) time and save the contents to a queue. Given the high-resolution capabilities of the LPC4088’s ADC, the system will be able to capture small variations in the signal, which will be crucial for accurately distinguishing between dots and dashes in Morse code. There are many ways to protect the analogue pin from overvoltage, such as using a 3.3V Zener diode or TVS diode in parallel with the output and some biasing resistors if voltage stepping down is necessary.

As for digital conversion, to determine the presence of the ASK signal, a Fourier transform algorithm will be used on the queue of analogue values to identify the frequency components. The presence of the frequency used in the ASK signal at a reasonable magnitude will be considered as a 1, and the lack of it, a 0. This value will be passed to the next stage responsible for Morse code translation.

To ensure accurate and consistent sampling intervals, the LPC4088’s SysTick timer will be utilized. This will trigger periodic interrupts, ensuring that the sampling happens at precise intervals and that short, unwanted pulses don’t affect the signal. The timer will be configured to trigger an ADC conversion at consistent time intervals, thus preserving the signal’s integrity for further processing.

\subsection{Morse Code Translation}
Due to the binary nature of Morse code, it allows the use of a binary tree structure where each node is reachable through a dot or a dash. As seen in the Morse code document \cite{itu2009}, it is intuitive that a tree of characters that are reachable using dot and dash traversals would allow for fast lookup and efficient memory usage due to the compact nature of characters, which are stored as 8-bit integers. A simplified binary tree which does not include all necessary symbols is shown in Figure 2 \cite{morsecode-tree}.

\begin{figure}[h]
    \centering
    \includegraphics[width=0.45\textwidth]{Morse-code-tree.png}
    \caption{Morse code binary tree. Note it is simplified and does not include all necessary character nodes that will be present in the full implementation.}
    \label{fig:morse_tree}
\end{figure}

Our implementation will extend this tree to include traversals to nodes that map to all characters or symbols in the Morse code document. The LPC4088's ARM Cortex-M4 processor will handle the traversal of the tree with minimal latency, thanks to its high clock speed and efficient execution of binary search algorithms.

The microcontroller will receive signals and, based on a defined unit length, will determine if it is a dot (of unit length) or a dash (of 3 unit lengths), and decode the signal using the binary tree. The efficiency of the tree traversal is key to ensuring the system responds quickly, decoding Morse code in O(log$_2$ n) time. This structure will be optimized to handle large inputs and scale effectively with increasing signal complexity.

\subsection{LCD Display}
To display the translated Morse code signal on the LCD, predefined functions from Lab 4 will be used. Functions such as \textit{lcd\_write\_data}, \textit{lcd\_put\_char}, and \textit{lcd\_print} allow us to project our Morse code translations to the LCD screen. The LPC4088 microcontroller supports multiple communication protocols (SPI, I2C, or parallel data) for interfacing with LCDs. In our case, we will be using SPI for data transmission due to its efficiency in sending data to the display.

The \textit{lcd\_put\_char} function is responsible for displaying individual characters, while \textit{lcd\_print} allows for printing entire strings. These functions utilize a serial-to-parallel shift register on the LPC4088, which minimizes the number of pins needed to 3 GPIO pins, optimizing hardware resource usage while maintaining efficient data communication with the LCD module. The functions in \textit{lcd.c} send data to the LCD in 4-bit mode, meaning any 8-bit character is split into two 4-bit chunks before being sent to the LCD. Scrolling or text-wrapping functionality could be added to ensure long outputs are sufficiently represented even if they exceed the LCD's dimensions.

\subsection{Scheduling}
As all functions depend on data provided by previous functions, out-of-order execution is inevitable to have a remotely efficient implementation. Thus, it is best that when a function finishes execution and generates new data (e.g., digital conversion), an interrupt is raised for the next function (Morse code translation) to run. Most functions aren’t time-critical and thus do not need high priority, except sampling, where the time between samples must be equal. Thus, if a sample is needed, it takes precedence.

The LPC4088’s interrupt controller will be used to handle scheduling tasks effectively. Interrupts will be raised at key stages in the signal processing pipeline, such as after each sample is taken or after the completion of the Morse code translation, triggering the next task. By leveraging interrupts, the system ensures efficient and responsive task management.

In addition, the Direct Memory Access (DMA) feature of the LPC4088 will be leveraged to handle efficient data transfer from the ADC to memory, reducing CPU overhead and ensuring faster data handling. The DMA will be crucial to maintaining real-time performance, especially when dealing with continuous signal sampling.

\section{Management}
Based on Fig. 1, the digital-to-character implementation has a robust approach regarding the Morse code binary tree, but an implementation to hide spaces is not yet finalized. However, this allows for dedicating more time to intensive stages, namely the analog-to-digital input conversion stage. Although the rest of the code still needs to be constructed, code at every stage will be designed with modularity in mind. It is important to note that the digital-to-char stage depends on the previous sampling and ASK-to-digital conversion stages.

The ASK-to-digital conversion stage is expected to be the most laborious, given sampling and digital-to-char conversion are relatively simple. Therefore, the current timeline will focus on implementing all four stages represented in Fig. 1 in parallel, with a higher consideration for the analog-to-digital stage, with a fully-fledged implementation expected around Easter. Finally, all functionalities will be seamlessly integrated using scheduling algorithms. This will take place within the final phases of development.

Because different stages require different skills, task distribution is accorded based on proficiency and capability in the given task, but it is important to note that each member will participate in an active role at all stages of development. The sampling stage will be distributed to Daniel based on his computer engineering capabilities. The digital conversion stage will be handled by all members as it seems to be the most challenging. Akram shall implement the Morse code translation via a binary tree for the digital-to-char conversion, whereas Antoni will be focused on integrating the LCD functionality with the underlying logic and hardware to provide accurate representations of outputs in accordance with the Morse code document standards.

The documentation for each stage in this brief and for any code will be written by the respective members, and each subsequent record of development for specific stages will be written by the assigned members. Unit tests will be developed for code at each stage, and members will meet on a regular basis each week to discuss ongoing development in particular stages and how they affect the Morse code receiver implementation more broadly.

\section{Closure}
The development of the Morse code receiver is structured into distinct stages, each focusing on critical aspects such as signal sampling, digital conversion, Morse code translation, and LCD display. By leveraging efficient algorithms, such as the binary tree traversal for Morse code decoding, and optimizing hardware usage with a serial-to-parallel shift register, the system is designed to be both robust and scalable.

Moving forward, the primary focus will be on finalizing the implementation of signal sampling and ASK-to-digital conversion, as these form the foundation for the entire decoding process. Task scheduling will be refined to ensure seamless communication between system components, and performance testing will be conducted to assess accuracy, noise resilience, and response time.

With a structured timeline and clear task distribution, the project is expected to progress efficiently toward the development of a functional Morse code receiver. The final implementation will be evaluated based on performance metrics, and necessary optimizations will be made to ensure reliability in real-world conditions.

\bibliographystyle{ieeetr}
\bibliography{references}

\begin{bibliography}

\end{bibliography}

\end{document}
